\documentclass[twoside,11pt]{article}
\usepackage{isipta}
%% Comment these two lines if you don't need UTF8
\usepackage[utf8]{inputenc}
\inputencoding{utf8} 
% Note that pgm.sty includes epsfig, amssymb, natbib and graphicx,
% and defines many common macros, such as 'proof' and 'example'.
% It also sets the bibliographystyle 
%% Put here the import commands of the packages you need and your custom commands 
\usepackage{hyperref,color,soul,booktabs,bm,algpseudocode,algorithm,tikz,pgfplots}
\usepackage{xargs}                      % Use more than one optional parameter in a new commands
%\usepackage[pdftex,dvipsnames]{xcolor}  % Coloured text etc.
% TIKZ for drawing
\usetikzlibrary{shapes.geometric} % for diamond nodes shape
\usepgflibrary{shapes.arrows} % for latex arrow shape
\newtheorem{mydef}[theorem]{Definition}
\newtheorem{myex}[theorem]{Example}
\setulcolor{blue}
\newcommand{\BibTeX}{\textsc{B\kern-0.1emi\kern-0.017emb}\kern-0.15em\TeX}
% Running title and authors 
\ShortHeadings{Online Variable Elimination for Credal Polytrees}{De Bock, Huber, and Antonucci}
%\pgmheading{1}{2000}{1-48}{4/00}{10/00}{Professorson and Teacherman}
%\firstpageno{10}
\usepackage{todonotes}
\presetkeys{todonotes}{fancyline, color=blue!30}{}
%\usepackage[colorinlistoftodos,prependcaption,textsize=tiny]{todonotes}
\begin{document}
% Title and authors
\title{Online Variable Elimination for Credal Polytrees}
\author{\name Jasper De Bock \email jasper.debock@ugent.be\\
\addr Ghent University, IDLab\\
Belgium\\
\AND
\name David Huber \email david@idsia.ch\\
\name Alessandro Antonucci \email alessandro@idsia.ch\\
\addr Istituto Dalle Molle di Studi Sull'Intelligenza Artificiale (IDSIA)\\
Switzerland}
\maketitle
\begin{abstract}%   <- trailing '%' for backward compatibility of .sty file
We propose a variable-elimination algorithm for exact inference in singly connected credal networks with strong independence. Compared to earlier works in this direction, the necessary extensive specification of the local models and the convexification are performed in an online way. This might produce considerable savings in memory and, for convexification only, time. Moreover, we show that with singly connected networks the convexification of the intermediate results of the inference can be achieved before the elimination step in a space with the same dimensionality of the variable to eliminate. This prevents issues related to convex hulls in high dimensions. Despite the NP-hardness of the inference even in the case of singly connected topologies, the experiments on a benchmark of randomly generated credal polytrees shows that exact inference can be achieved in models up to xxx variables if the number xxx. A further speed-up, at the price of an approximation, is achieved by rounding the values of the extreme points before the convexification, while an extension to general topologies can be achieved by loopset conditioning.
\end{abstract}
\begin{keywords}
Credal networks; convex hull; variable elimination; cutset conditioning.
\end{keywords}
\section{Introduction}\label{sec:intro}
Bayesian networks \citep[BNs]{2009:koller} are the most popular class of multivariate probabilistic models based on directed graphs. These tools are subject of intense theoretical investigation and their applications are ubiquitous, e.g., \cite{landuyt2013,de2013}. A common inference task with BNs, called \emph{updating}, consists in the computation of the probabilities of a variable of interest possibly given an instantiation of some other variables. BNs updating is a NP-hard task \citep{cooper1990}, which might become tractable under some additional assumption. If the standard \emph{variable elimination} \citep[VE]{kohlas} scheme is adopted, both time and space complexity are exponential in the \emph{treewidth}, which roughly describes how much a directed graph is similar to a singly-connected model (also called a polytree). Efficient updating can be therefore achieved (among others) in polytree-shaped BNs. This is basically equivalent to Pearl's message-propagation in polytrees \cite{pearl}.

Credal networks \citep[CNs]{cozman2000a} are a generalization of BNs based on the imprecise probability theory \citep{walley1991}. As well as a BN is specified by \emph{local} (w.r.t to the graph) probability mass functions, in a CN these are replaced by convex sets of them. We consequently call \emph{credal set updating} the task of  computing the credal set made of all the posterior mass functions for the queried variable given the evidence computed for all the BNs consistent with the CN specification. Here we call this task credal-set updating, while most of the authors intend updating as the computation of the lower and upper bounds of the posterior probability with respect to the CN specification. To avoid confusion, we call this latter task \emph{probabilistic updating} (PU). 

Being generalisations of their Bayesian counterpart, both tasks are NP-hard. Yet, unlike the case of BNs, PU remains NP-hard even in polytree-shaped CNs \citep{maua2014}, while an efficient scheme has been developed only for the special case of polytrees with all the variables binary \citep{fagiuoli1998a}. Note that for binary queries PU and CU coincides.\footnote{A credal set over a binary variable cannot have more thant two xxx.}

Most of the literature about inference in CN is focused on approximate procedures for PU \cite{xxx}. Two remarkable exception are the early attempt of \cite{rocha} and a more recent paper by \cite{maua} where CU by VE is considered. In the first case, the convexification step, which is crucial to keep a bound on the exponential explosion on the number of extreme points, is performed. Unfortunately, as we show later, this induces an outer approximation. To avoid the outer approximation and keep the inferences exact, the second paper assumes the models extensively specified. In the present paper, we stick on the second approach, but assuming that the original model is separately specified. We therefore develop a number of online procedure which allow to perform the algebraic operations required by inference without xxx. Moreover, we suggest to perform convexification only on the space of single variables xxx.

\section{Basics}\label{sec:basics}
\subsection{Bayesian and credal networks (BNs \& CNs)}
Given a discrete variable $X$, $\mathcal{X}$ denotes the set of its possible values and $x$ a generic element of this set. $P(X)$ is a probability mass function (PMF) over $X$ and $P(x)$ the probability assigned to the atom $x\in\mathcal{X}$. $K(X)$ denotes instead a set of PMFs over $X$, which is also called \emph{credal set} (CS). The bounds of the expectation of a linear function of $X$ w.r.t. $K(X)$ are not affected by the removal of the inner points of $K(X)$. We denote as $\overline{K}(X)$ the result of this operation. Here we focus on CSs with a finite number of extreme (i.e., non inner) points.


Given a set of variables $\bm{X}:=(X_0,X_1,\ldots,X_n)$, a BN is a pair composed by a directed acyclic graph $\mathcal{G}$ whose nodes are in one-to-one correspondence with the elements of $\bm{X}$ and a collection of PMFs $P(X_i|x_{\pi_i})$, one for each value of $x_{\pi_i}$, for each $X_i \in \bm{X}$, with $\Pi_i$ set of variables corresponding to the parents (i.e., predecessors according to $\mathcal{G}$) of $X_i$. In a CN specification, each $P(X_i|x_{\pi_i})$ is simply replaced by a CS $K(X_i|x_{\pi_i})$.

A BN defines a joint PMF $P(\bm{X})$, which factorises as \begin{equation}\label{eq:joint}
P(\bm{x}) = \prod_{i=0}^n P(x_i|\pi_i),
\end{equation}
for each $\bm{x}\in\mathcal{\bm{X}}$ and where the values of $x_i$ and $x_{\pi_i}$ are those consistent with $\bm{x}$. Similarly, a CN defines a joint CS $K(\bm{X})$ whose elements are PMFs obtained as in Eq.~\ref{eq:joint} with $P(X_i|\pi_i)\in K(X_i|\pi_i)$. 

\subsection{Standard, Credal, and Bound Udating}

Consider a BN and a CN over $\bm{X}$. Without lack of generality we can assume that the variable of interest $X_0$. Given a set of variables $X_E$, denotes as $x_E\in\mathcal{X}_E$ an instance of these variables. Updating a BN is intended as the computation of $P(x_0,x_E)$, obtained by summing out the variables in $\bm{X}\setminus \{X_0,X_E\}$ from Eq.~\ref{eq:joint}, for each $x_0\in\mathcal{X}_0$, from which $P(X_0|x_E)$ can simply obtained by normalisation, provided that $P(x_E)>0$.

We can analogously intend updating in credal networks as the computation of the set $K(X_0,x_E)$, from which, by elementwise normalisation we obtain the updated CS $K(X_0|x_E)$. We call this CN inference task credal updating (CU) and distinguish it from the computation of the bounds $\underline{P}(x_0|x_E)$ and $\overline{P}(x_0|x_E)$ which is called here \emph{bound updating}.

\subsection{Variable Elimination (VE) in BNs}

VE is a standard approach to PU in BNs. The idea is to perform marginalisation only over the combination of local models including the variable to eliminate.

Given two (possibly joint )variables $X_I$ and $X_J$, a conditional probability table (CPT) $P(X_I|X_J)$ is a collection of PMF over $X_I$, one for each $X_J\in\mathcal{X}_J$. Given twp CPTs $P(X_I|X_J)$ and $P(X_K|X_L)$ we can simply obtain a new CPT, denoted as $P(X_I|X_J)\otimes P(X_K|X_L)$ by combination. The combined CPT is defined over $X_I \cup X_L$ given $X_J \cup X_M \setminus \{ X_I \cup X_J\}$. Given a CPT $P(X_I,X_J|X_K)$
the marginalization of $X_J$ is denoted as $\sum_{X_j} P)(X_I|X_K)$ being simply a CPT $P(X_I|X_K)$ obtained by summing out $X_J$ on each conditional PMF. Finally, given a CPT $P(X_I,X_J|X_K)$ and a state $x_J\in\mathcal{X}_J$ we cal focusing a CPT $P(X_I,x_j|X_K)$. Note that strictly speaking this is not a CPT over $X_I$ because of the non-normalisation w.r.t. $X_I$.

With this simple algebraic structure (see \cite{kohlas2003} for a xxx), we formulate the VE algorithm as in Table X.


\begin{algorithm}
	\caption{VE algorithm
		%\vskip 0.6mm [Parameters] $s$ (maximum number of no-improve iterations) and $t$ (number of restarts) 
		\vskip 0.6mm [INPUT] a BN specification, an evidence $(x_{n-n_e},\ldots,x_n)$%\\credal network specification $\{ K(X_i|\pi_i) \}_{i=0,\ldots,n}^{\pi_i\in\Omega_{\Pi_i}}$
		\vskip 0.6mm [OUTPUT] $K(X_o|x_{n-n_e},\ldots,x_n)$} %an upper approximation of $\underline{P}(x_0)$\label{algo:glp}}
	\begin{algorithmic}[1]
		%\State $pp \gets 1.0$
		\State $\Phi := \{ P(X_i|\Pi_i) \}_{i=0}^n$ 
		\For{$k \gets n,\ldots,0$}%\Comment{random restarts}
		\State $\phi_k =1$ %$\tilde{P}(X_i|\pi_i)$ $\gets$ randomly pick from $\mathrm{ext}[K(X_i|\pi_i)]$ $\forall i,\pi_i$ \Comment{initialization}
		\For{$\phi \in \Phi$}
		\If{$X_k$ is in the argument of $\phi$}
		\State $\phi_k \gets \phi_k \otimes \phi$
		\State $\Phi \gets \Phi \setminus \{ \phi \}$
		\EndIf
		\EndFor
		\EndFor
		\If{$k = 0 | E$}
		\State $\phi_k \gets \sum_{X_k} \phi_k $
		\EndIf
		\State $\Phi \gets \Phi \cup \{ \sum_{X_k} \phi_k \}$
		\State {\bf return} $pp$ 
	\end{algorithmic}
\end{algorithm}





To perform PU in a BN
here the right-hand side simply follows from the factorisation in \cite{x} the notion marginalisation of the unqueried and unobserved variable and the Bayes' theorem. For Bayesian networks updating is intended the identification of the lower/upper bounds.

Variable elimination in a BN can be easily achieved as follows xxx. Given the ord

\begin{verbatim}
For X in X_n ... X_1
Collect all the CPTs including X in B
combine \otimes B
if X is not observed: sum-out X
else do focusing on xxx
\end{verbatim}
When the procedure end the algorithm returns $P(X_0,x_E)$ which can be used to compute $P(X_0|x_E)$ by a simple marginalization. The combination operation simply consists in the product of two CPTs, marginalisation simply coincides with the sum of a variable on the left and focusing with the xxx.

\section{Credal Variable Elimination (dummy)}
To extend the procedure outlined in the previous section we simply need to extend the operations xx.

\subsection{The Algebra of Credal Tables}
Given two variables $X$ and $Y$, we use notation $\{ K(X|y) \}_{y\in\mathcal{Y}}$ to denote a collection of (conditional) CSs over $X$, one for each value of $Y$. Given this object, we denote as $K(X|Y)$ the collection of all the probability tables $P(X|Y)$ obtained by combining, separately for each $y$, the extreme points xxx. We call this operation \emph{extensivisation} of the collection of conditional credal sets and cradal tables the corresponding object.

Given two credal tables $K(X|Y)$ and $K(W|Z)$ we define the following combination operator. $K(X|Y) \otimes K(W|Z)$ is a credal table $K(X \cup W| Y,Z )$ obtained by all the possible combinations. 

Given a credal table $K(X,Y|Z)$ the marginalization is xxxx. 

Similarly the focusing operation consistst in $K(X,y|Z)$. Alternatively if the focussing is on $Z$, we have $K(X,Y|z)=K'(X,Y)$

Finally the convexification of a CT $K(X|Y)$ is intended as the convex hull operation xxx. 

After all these operations we can reformulate variable elimination in credal case as follows

\begin{verbatim}
For X in X_n .... X_1
Collect all the credal tables including X in B
If not already done, perform extensive all the elements
combine them with \otimes B
Take the convex hull
marginalise out X
Take the convex hull
\end{verbatim}
The procedure ends up with $K(X_q,x_E)$ whose vertex can be normalised to obtain $K(X_q|x_E)$ (this corresponds to a practical implementation of the generalised Bayes theorem \cite{walley}).

Let us consider the following task. Given a queried variable $X_0$, and a set of evidences, say $(x_{n-e},\ldots,x_n)$ of the corresponding variables, we obtain $K(X_0,x_{n-e},\ldots,x_e)$ as follows.

\begin{algorithm}
\caption{VE algorithm
%\vskip 0.6mm [Parameters] $s$ (maximum number of no-improve iterations) and $t$ (number of restarts) 
\vskip 0.6mm [INPUT] a CN specification, an evidence $(x_{n-n_e},\ldots,x_n)$%\\credal network specification $\{ K(X_i|\pi_i) \}_{i=0,\ldots,n}^{\pi_i\in\Omega_{\Pi_i}}$
\vskip 0.6mm [OUTPUT] $K(X_o|x_{n-n_e},\ldots,x_n)$} %an upper approximation of $\underline{P}(x_0)$\label{algo:glp}}
\begin{algorithmic}[1]
%\State $pp \gets 1.0$
\State $\Phi := \{ P(X_i|\Pi_i) \}_{i=0}^n$ 
\For{$k \gets n,\ldots,0$}%\Comment{random restarts}
\State $\phi_k =1$ %$\tilde{P}(X_i|\pi_i)$ $\gets$ randomly pick from $\mathrm{ext}[K(X_i|\pi_i)]$ $\forall i,\pi_i$ \Comment{initialization}
\For{$\phi \in \Phi$}
\If{$X_k$ is in the argument of $\phi$}
\State $\phi_k \gets \phi_k \otimes \phi$
\State $\Phi \gets \Phi \setminus \{ \phi \}$
\EndIf
\EndFor
\EndFor
\If{$k = 0 | E$}
\State $\phi_k \gets \sum_{X_k} \phi_k $
\EndIf
\State $\Phi \gets \Phi \cup \{ \sum_{X_k} \phi_k \}$
\State {\bf return} $pp$ 
\end{algorithmic}
\end{algorithm}

\section{Online Variable Elimination}
\subsection{The case of polytrees}

The complexity of both the procedures in xxx and xxx depends on the selected optimisation order and the topology of the network. The treewidth is the maximum arity of xxx. A standard moralisation approach leads to xxx. In the special case of polytree-shaped topology (as well as in the more general case of bounded treewidth), whose treewith is one, the approach has polynomial space and time complexity, the treewidth corresponding to xxx. Yet, in the credal case, this is not sufficient to have a polynomial inference algorithm. Even if the arity of the maximuma credal table is bounded, the memory and the time complexity can explode because of a huge number of extreme tables. 

Explain marginals in binary polytrees xxx.

\subsection{Online hull and extensivisation}
To speed up the procedure let's.


\subsection{A demonstrative example}
To illustrate the different


\section{Extending the algorithm}
\subsection{Rounding the potentials}
The above considered procedure can be used to credal updating in polytree shaped CNs. Yet the problem, even in the case of marginal computation. Consider for instance the problem used by xxx to prove that marginal inference in ternary polytrees is NP-hard. Following the directions in xxx, we achieve an approximated.
\subsection{Multiply-connected networks}
Yet, it is not applicable to multiply connected models and, with even with simple polytrees it might

\subsection{A demonstrative example}\label{sec:example}
To illustrate the different


\begin{figure}
\centering
\begin{tikzpicture}[semithick]
\tikzset{ every node/.style={circle, draw, fill=white, minimum size=16pt, inner sep=1pt}}
\tikzstyle{note}=[draw=white,fill=white,font=\small,circle,inner sep=0pt]
\tikzstyle{evidence}=[fill=gray!24]
\tikzstyle{query}=[fill=black!80, text=white]
\tikzstyle{ed}=[draw=black,line width=.8pt, postaction={decorate}, decoration={markings,mark=at position 1.0 with {\arrow[draw=black,line width=.8pt]{>}}}]
%\begin{scope}
% vertices
\node (a) at (0,1) {};
\node (b) at (1,1) {};
\node (c) at (2,1) {};
\node[query] (d) at (3,1) {0};
\node[evidence] (e) at (-.5,0) {3};
\node[evidence] (f) at (0.5,0) {4};
\node (g) at (2,0) {};
\node[evidence] (h) at (3.5,0) {};
\node (m) at (2.75,-1) {};
\node[evidence] (n) at (2,-2) {};
\node[] (o) at (1.5,2) {};
\node[] (p) at (.5,2) {};
%\node[note] (label) at (1,-1) {(a)};
\draw[->] (a) -- (e);
\draw[->] (a) -- (f);
\draw[->] (b) -- (f);
\draw[->] (b) -- (g);
\draw[->] (c) -- (g);
\draw[->] (d) -- (g);
\draw[->] (d) -- (h);
\draw[->] (g) -- (m);
\draw[->] (m) -- (n);
\draw[->] (o) -- (b);
\draw[->] (p) -- (b);
%\end{scope}
%\begin{scope}[xshift=5cm]
%\end{scope}
\end{tikzpicture}
\caption{A polytree-shaped credal network}
\label{fig:polytree}
\end{figure}







%\input{experiments}
\section{Extending the algorithm}
\subsection{Rounding the potentials}
The above considered procedure can be used to credal updating in polytree shaped CNs. Yet the problem, even in the case of marginal computation. Consider for instance the problem used by xxx to prove that marginal inference in ternary polytrees is NP-hard. Following the directions in xxx, we achieve an approximated.
\subsection{Multiply-connected networks}
Yet, it is not applicable to multiply connected models and, with even with simple polytrees it might
\section{Experiments}
In order to demonstrate our algorithm we consider a benchmark of xxx. Note that unlike the experiments of xxx, we consider separately specified credal networks which are extensivised as in xxx when necessary (see line xxx of alg xx).
\section{Conclusions and Outlooks}
Let me see. Mention that we might explore other (more complex) inference task such as the computation of the probability of evidence as well as the maximum a posteriori hypothesis task (MAP), which in the credal setting can be formulated in different ways.
\bibliography{biblio}
\end{document}
%When the procedure end the algorithm returns $P(X_0,x_E)$ which can be used to compute $P(X_0|x_E)$ by a simple marginalization. The combination operation simply consists in the product of two CPTs, marginalisation simply coincides with the sum of a variable on the left and focusing with the xxx.
%For sets of conditional mass functions we use the short notation $P(X_I|X_J)=\{ P(X_I|x_J) \}_{x_J}$. Such a collection of mass functions can be clearly regarded as a real-valued functions of either the variables on the left and the right of the conditioning bar. For CSs we analougously use notation $K(X_I|X_J)$ for a collection of credal sets. Yet, in this case the position wrt conditioning bar is not irrelevant. From an algebraic point of view this is a set-valued function of $X_I$ separately for each value of $X_J$. To bypass this issue it is possible to formulate each local specification of conditional . This is for instance the approach xxx. We use notation $K(X_I||X_J)$ to denote an extensive formulation of $K(X_I|X_J)$. Note that if on a side the extensive specification allows for xxx, it might be highly redundant xxx.
%Variable elimination is a standard inference strategy for xxx.
%Let us first explain how the computation of a marginal (i.e., $X_E = \emptyset$). Without lack of generality let us assume that the queried variable is $X_0$. 
%Chain and reversed chain
%\begin{myex}
%Let $X$ and $Y$ be two Boolean variables. If $P(X|Y=0)=\mathrm{CH}[]$ and $K(X|Y=1)=$ Then the xxx.
%\end{myex}
%\section{Cutset conditioning}
%The VE procedure described in the previous section (whose worst case complexity remains exponentials) can be applied to polytree only. In order to extends these ideas to multiply connected topologies we adopt the same strategy considered for credal networks and based on the notion of cutset. It is a well-known fact that the arcs leaving an observed node in a Bayesian networks can be removed, provided that the local models of its children are replaced to xxx. This results have been shown to be valid for credal networks with strong independence by xxx. The cutset conditioning idea simply consists in detecting a set of nows whose observations. For Bayesian networks this procedure, whose complexity is xxx.
%Just notice that $P(X_q,x_E)$ shoul be eventually normalised in order to obtain $P(X_q|x_E)$.
%From the point of view of variable elimination, there is a crucial differ
%\section{Variable elimination with credal networks}
%The combination operator $\otimes$ is achieved by simple elementwise multiplication. As noticed by Koller, this combination does not necessarily produce a set of conditional mass function.
%To extend the VE algorithm to the CNs framework we need to extend to the credal sets framework the above considered operations of combination and marginalization. This is done in the following definitions.
%\begin{mydef}
%	Let $K(X_I|X_J)$ and  denote a collection of credal is denotes as $K(X||Y)$. We call this transformation extensivisation.
%\end{mydef}
%Consider an updating task $\underline{P}(x_q|x_E)$ in a credal network. If the network is not singly connected we detect a set of cutset conditioning $X_F$. Thus we compute $K(X_q,x_E,x_F)$ and hence $K(X_q,x_E|x_F)$.
%\[ \underline{P}(x_q|x_E) \leq \sum_{x_F} \underline{P}(x_q,x_F|x_E) \]
%\section{Experiments}
%\subsection{Randomly generated credal sets}
%\subsection{The benchmark}
%\subsection{Results}
%Exact BNs inference is NP-hard task in general. Network topology is primarly setting this complexity level. In fact polytree shaped BNs can be updated in polynomial time (provided the the indegree is bounded), while general models can be only xx. CNs extend BNs by allowing for set-valued specification of the local models, with single conditional probability mass functions in the conditional probability tables replaced by convex sets of them. Updating in these cases is intended as the computation of the lower and upper bounds of the posterior probability with respect. This is clearly still a hard task extending BNs inference. Exact inference in credal networks is considerably more difficult. A recent paper. A possible rationale about that is the fact also the vertices are involved in that.
%%%%%%%%%%%
%\appendix
%\acks{We would like to acknowledge support for this project from the Random Science Foundation.}
%\section{Proofs}
%\label{app:theorem}
%In this appendix we present a random filler.
%\vskip 0.2in
%\end{document}
