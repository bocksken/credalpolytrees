\section{Basics}\label{sec:basics}
\subsection{Bayesian and credal networks (BNs \& CNs)}
Given a discrete variable $X$, $\mathcal{X}$ denotes the set of its possible values and $x$ a generic element of this set. $P(X)$ is a probability mass function (PMF) over $X$ and $P(x)$ the probability assigned to the atom $x\in\mathcal{X}$. $K(X)$ denotes instead a set of PMFs over $X$, which is also called \emph{credal set} (CS). The bounds of the expectation of a linear function of $X$ w.r.t. $K(X)$ are not affected by the removal of the inner points of $K(X)$. We denote as $\overline{K}(X)$ the result of this operation. Here we focus on CSs with a finite number of extreme (i.e., non inner) points.


Given a set of variables $\bm{X}:=(X_0,X_1,\ldots,X_n)$, a BN is a pair composed by a directed acyclic graph $\mathcal{G}$ whose nodes are in one-to-one correspondence with the elements of $\bm{X}$ and a collection of PMFs $P(X_i|x_{\pi_i})$, one for each value of $x_{\pi_i}$, for each $X_i \in \bm{X}$, with $\Pi_i$ set of variables corresponding to the parents (i.e., predecessors according to $\mathcal{G}$) of $X_i$. In a CN specification, each $P(X_i|x_{\pi_i})$ is simply replaced by a CS $K(X_i|x_{\pi_i})$.

A BN defines a joint PMF $P(\bm{X})$, which factorises as \begin{equation}\label{eq:joint}
P(\bm{x}) = \prod_{i=0}^n P(x_i|\pi_i),
\end{equation}
for each $\bm{x}\in\mathcal{\bm{X}}$ and where the values of $x_i$ and $x_{\pi_i}$ are those consistent with $\bm{x}$. Similarly, a CN defines a joint CS $K(\bm{X})$ whose elements are PMFs obtained as in Eq.~\ref{eq:joint} with $P(X_i|\pi_i)\in K(X_i|\pi_i)$. 

\subsection{Standard, Credal, and Bound Udating}

Consider a BN and a CN over $\bm{X}$. Without lack of generality we can assume that the variable of interest $X_0$. Given a set of variables $X_E$, denotes as $x_E\in\mathcal{X}_E$ an instance of these variables. Updating a BN is intended as the computation of $P(x_0,x_E)$, obtained by summing out the variables in $\bm{X}\setminus \{X_0,X_E\}$ from Eq.~\ref{eq:joint}, for each $x_0\in\mathcal{X}_0$, from which $P(X_0|x_E)$ can simply obtained by normalisation, provided that $P(x_E)>0$.

We can analogously intend updating in credal networks as the computation of the set $K(X_0,x_E)$, from which, by elementwise normalisation we obtain the updated CS $K(X_0|x_E)$. We call this CN inference task credal updating (CU) and distinguish it from the computation of the bounds $\underline{P}(x_0|x_E)$ and $\overline{P}(x_0|x_E)$ which is called here \emph{bound updating}.

\subsection{Variable Elimination (VE) in BNs}

VE is a standard approach to PU in BNs. The idea is to perform marginalisation only over the combination of local models including the variable to eliminate.

Given two (possibly joint )variables $X_I$ and $X_J$, a conditional probability table (CPT) $P(X_I|X_J)$ is a collection of PMF over $X_I$, one for each $X_J\in\mathcal{X}_J$. Given twp CPTs $P(X_I|X_J)$ and $P(X_K|X_L)$ we can simply obtain a new CPT, denoted as $P(X_I|X_J)\otimes P(X_K|X_L)$ by combination. The combined CPT is defined over $X_I \cup X_L$ given $X_J \cup X_M \setminus \{ X_I \cup X_J\}$. Given a CPT $P(X_I,X_J|X_K)$
the marginalization of $X_J$ is denoted as $\sum_{X_j} P)(X_I|X_K)$ being simply a CPT $P(X_I|X_K)$ obtained by summing out $X_J$ on each conditional PMF. Finally, given a CPT $P(X_I,X_J|X_K)$ and a state $x_J\in\mathcal{X}_J$ we cal focusing a CPT $P(X_I,x_j|X_K)$. Note that strictly speaking this is not a CPT over $X_I$ because of the non-normalisation w.r.t. $X_I$.

With this simple algebraic structure (see \cite{kohlas2003} for a xxx), we formulate the VE algorithm as in Table X.


\begin{algorithm}
	\caption{VE algorithm
		%\vskip 0.6mm [Parameters] $s$ (maximum number of no-improve iterations) and $t$ (number of restarts) 
		\vskip 0.6mm [INPUT] a BN specification, an evidence $(x_{n-n_e},\ldots,x_n)$%\\credal network specification $\{ K(X_i|\pi_i) \}_{i=0,\ldots,n}^{\pi_i\in\Omega_{\Pi_i}}$
		\vskip 0.6mm [OUTPUT] $K(X_o|x_{n-n_e},\ldots,x_n)$} %an upper approximation of $\underline{P}(x_0)$\label{algo:glp}}
	\begin{algorithmic}[1]
		%\State $pp \gets 1.0$
		\State $\Phi := \{ P(X_i|\Pi_i) \}_{i=0}^n$ 
		\For{$k \gets n,\ldots,0$}%\Comment{random restarts}
		\State $\phi_k =1$ %$\tilde{P}(X_i|\pi_i)$ $\gets$ randomly pick from $\mathrm{ext}[K(X_i|\pi_i)]$ $\forall i,\pi_i$ \Comment{initialization}
		\For{$\phi \in \Phi$}
		\If{$X_k$ is in the argument of $\phi$}
		\State $\phi_k \gets \phi_k \otimes \phi$
		\State $\Phi \gets \Phi \setminus \{ \phi \}$
		\EndIf
		\EndFor
		\EndFor
		\If{$k = 0 | E$}
		\State $\phi_k \gets \sum_{X_k} \phi_k $
		\EndIf
		\State $\Phi \gets \Phi \cup \{ \sum_{X_k} \phi_k \}$
		\State {\bf return} $pp$ 
	\end{algorithmic}
\end{algorithm}





To perform PU in a BN
here the right-hand side simply follows from the factorisation in \cite{x} the notion marginalisation of the unqueried and unobserved variable and the Bayes' theorem. For Bayesian networks updating is intended the identification of the lower/upper bounds.

Variable elimination in a BN can be easily achieved as follows xxx. Given the ord

\begin{verbatim}
For X in X_n ... X_1
Collect all the CPTs including X in B
combine \otimes B
if X is not observed: sum-out X
else do focusing on xxx
\end{verbatim}
When the procedure end the algorithm returns $P(X_0,x_E)$ which can be used to compute $P(X_0|x_E)$ by a simple marginalization. The combination operation simply consists in the product of two CPTs, marginalisation simply coincides with the sum of a variable on the left and focusing with the xxx.
