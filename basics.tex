\section{Basics}\label{sec:basics}
\subsection{Bayesian and credal networks (BNs \& CNs)}
Given a discrete variable $X$, $\mathcal{X}$ denotes the set of its possible values and $x$ a generic element of this set. $P(X)$ is a probability mass function (PMF) over $X$ and $P(x)$ the probability assigned to the atom $x\in\mathcal{X}$. $K(X)$ denotes instead a set of PMFs over $X$, which is also called \emph{credal set} (CS). The bounds of the expectation of a linear function of $X$ w.r.t. $K(X)$ are not affected by the removal of the inner points of $K(X)$. We denote as $\overline{K}(X)$ the result of this operation. Here we focus on CSs with a finite number of extreme (i.e., non inner) points.


Given a set of variables $\bm{X}:=(X_0,X_1,\ldots,X_n)$, a BN is a pair composed by a directed acyclic graph $\mathcal{G}$ whose nodes are in one-to-one correspondence with the elements of $\bm{X}$ and a collection of PMFs $P(X_i|x_{\pi_i})$, one for each value of $x_{\pi_i}$, for each $X_i \in \bm{X}$, with $\Pi_i$ set of variables corresponding to the parents (i.e., predecessors according to $\mathcal{G}$) of $X_i$. In a CN specification, each $P(X_i|x_{\pi_i})$ is simply replaced by a CS $K(X_i|x_{\pi_i})$.

A BN defines a joint PMF $P(\bm{X})$, which factorises as \begin{equation}\label{eq:joint}
P(\bm{x}) = \prod_{i=0}^n P(x_i|\pi_i),
\end{equation}
for each $\bm{x}\in\mathcal{\bm{X}}$ and where the values of $x_i$ and $x_{\pi_i}$ are those consistent with $\bm{x}$. Similarly, a CN defines a joint CS $K(\bm{X})$ whose elements are PMFs obtained as in Eq.~\ref{eq:joint} with $P(X_i|\pi_i)\in K(X_i|\pi_i)$. 

\subsection{Bayesian (U), Credal (CU), and Bound Udating(BU)}
Consider a BN and a CN over $\bm{X}$. Without lack of generality we can assume that the variable of interest $X_0$. Given a set of variables $X_E$, denotes as $x_E\in\mathcal{X}_E$ an instance of these variables. Updating a BN is intended as the computation of $P(x_0,x_E)$, obtained by summing out the variables in $\bm{X}\setminus \{X_0,X_E\}$ from Eq.~\ref{eq:joint}, for each $x_0\in\mathcal{X}_0$, from which $P(X_0|x_E)$ can simply obtained by normalisation, provided that $P(x_E)>0$.

We can analogously intend updating in credal networks as the computation of the set $K(X_0,x_E)$, from which, by elementwise normalisation we obtain the updated CS $K(X_0|x_E)$. We call this CN inference task credal updating (CU) and distinguish it from the computation of the bounds $\underline{P}(x_0|x_E)$ and $\overline{P}(x_0|x_E)$ which is called here \emph{bound updating}.

\subsection{Variable Elimination (VE) in BNs}
VE is a standard approach to SU in BNs which consists in performing the marginalisation only over the combination of local models including the variable to eliminate. This is based on three simple operations over conditional probability tables.

A conditional probability table (CPT) $P(X_I|X_J)$ over $X_I$ given $X_J$ is a collection of PMFs over $X_I$, one for each $X_J\in\mathcal{X}_J$. Given two CPTs $P(X_I|X_J)$ and $P(X_K|X_L)$, their \emph{combination} $P(X_I|X_J)\otimes P(X_K|X_L)$ is a CPT over $X_I \cup X_K$ given $\{X_J \cup X_L\} \setminus \{ X_I \cup X_K\}$ obtained by simple product *** say more? ***. Given a CPT $P(X_I,X_J|X_K)$, \emph{summing out} $X_J$ from this CPT defines a CPT over $X_I$ given $X_K$, denoted as $\sum_{X_j} P)(X_I|X_K)$, and obtained by summing out $X_J$ from each conditional PMF. Finally, given a CPT $P(X_I,X_J|X_K)$ and a state $x_J\in\mathcal{X}_J$ the \emph{focusing} of this CPT on $x_J$ is a CPT over $X_I$ given $X_K$, denoted as $P(X_I,x_j|X_K)$, and obtained by considering only the probabilities for $X_J=x_J$.\footnote{Strictly speaking the result of a focusing is not a CPT as the model is not normalised.}


With this simple algebraic structure (see \cite{kohlas2003} for a more deeper discussion), we formulate the VE algorithm as in Algorithm X.


\begin{algorithm}
\caption{VE algorithm
\vskip 0.6mm [INPUT] BN specification, query $X_q$, evidence $X_E=x_E$ \vskip 0.6mm [OUTPUT] $P(X_0|x_E)$
\label{alg:VEBN}}
\begin{algorithmic}[1]
%\State $pp \gets 1.0$
\State $\Phi := \{ P(X_i|\Pi_i) \}_{i=0}^n$
\Comment{All CPTs in input}
\For{$k \gets n,\ldots,0$}
\Comment{Loop over the variables (to be eliminated)}
\State $\phi_k =1$ 
\Comment{Initialise the combined CPT}
\For{$\phi \in \Phi$}
\If{$X_k$ is in the argument of $\phi$}
\Comment{Select the CPTs containing $X_k$}
\State $\phi_k \gets \phi_k \otimes \phi$
\Comment{Combination}
\State $\Phi \gets \Phi \setminus \{ \phi \}$
\EndIf
\EndFor
\If {$X_k \not\in \{ X_E \cup X_q\}$}
\State $\phi_k \gets \sum_{X_k} \phi_k$
\Comment{Sum out unobserved/unqueried variables}
\EndIf
\If {$X_k \in X_E$}
\State $\phi_k \gets \phi_k(\cdot,x_k)$
\Comment{Focus for observed variables ($x_k$ consistent with $x_E$)}
\EndIf
\State $\Phi \gets \Phi \cup \{ \phi_k \}$
\EndFor
\State $P(X_q,x_E) \gets \Phi$ 
\Comment{$\Phi$ now includes a single element}
\State {\bf return} $\frac{P(X_q,x_E)}{\sum_{X_q} P(X_q,x_E)}$
\Comment{Conditioning by normalisation (provided that $P(x_E)>0)$}

	\end{algorithmic}
\end{algorithm}



When the procedure end the algorithm returns $P(X_0,x_E)$ which can be used to compute $P(X_0|x_E)$ by a simple marginalization. The combination operation simply consists in the product of two CPTs, marginalisation simply coincides with the sum of a variable on the left and focusing with the xxx.
