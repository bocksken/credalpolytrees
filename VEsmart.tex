\section{Online Variable Elimination}
\subsection{The case of polytrees}

The complexity of both the procedures in xxx and xxx depends on the selected optimisation order and the topology of the network. The treewidth is the maximum arity of xxx. A standard moralisation approach leads to xxx. In the special case of polytree-shaped topology (as well as in the more general case of bounded treewidth), whose treewith is one, the approach has polynomial space and time complexity, the treewidth corresponding to xxx. Yet, in the credal case, this is not sufficient to have a polynomial inference algorithm. Even if the arity of the maximuma credal table is bounded, the memory and the time complexity can explode because of a huge number of extreme tables. 

Explain marginals in binary polytrees xxx.

\subsection{Online hull and extensivisation}
To speed up the procedure let's.


\subsection{A demonstrative example}
To illustrate the different


\section{Extending the algorithm}
\subsection{Rounding the potentials}
The above considered procedure can be used to credal updating in polytree shaped CNs. Yet the problem, even in the case of marginal computation. Consider for instance the problem used by xxx to prove that marginal inference in ternary polytrees is NP-hard. Following the directions in xxx, we achieve an approximated.
\subsection{Multiply-connected networks}
Yet, it is not applicable to multiply connected models and, with even with simple polytrees it might
