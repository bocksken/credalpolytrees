\section{Introduction}\label{sec:intro}
Bayesian networks \citep[BNs,][]{2009:koller} are the most popular class of multivariate probabilistic models based on directed graphs. These tools are subject of intense theoretical investigation and their applications ubiquitous, \citep[e.g.,][]{landuyt2013,de2013}. A common BN inference task, called \emph{updating}, consists in the computation of the probabilities of a queried variable given an instantiation of some other variables. BNs updating is a NP-hard task \citep{cooper1990}, which might become tractable under some additional assumption. Given an ordering of the variables, the \emph{variable elimination} \citep[VE,][]{zhang1994simple} is a standard procedure to update BNs. Yet, VE complexity depends on the ordering, and finding the optimal order is in its turn NP-hard. This is not the case with singly connected topologies (also called a polytrees, see, e.g., Fig.~\ref{fig:polytree}), for which an optimal ordering, which takes complexity exponential in the maximum number of parents, can be easily found in linear time.%\todo[inline]{Mention Pearl's algorithm too?}

Credal networks \citep[CNs,][]{cozman2000a} are a generalization of BNs based on the imprecise probability theory \citep{walley1991}. While a BN is specified by local (w.r.t. the graph) probability mass functions,these are replaced by sets of mass functions (also called \emph{credal} sets) in CNs. We call \emph{credal set updating} (CU) the task of computing the credal set of all the mass functions for the queried variable given the evidence computed in all the BNs consistent with the CN specification (i.e., with the local credal sets). Most of the authors intend CN updating as the computation of the lower and upper bounds of the posterior probability with respect to the CN specification. We call this latter task \emph{probabilistic updating} (PU). Being generalisations of their Bayesian counterpart, both tasks are NP-hard. Yet, unlike the case of BNs, PU remains NP-hard even in polytree-shaped CNs \citep{maua2014}, while an efficient exact scheme has been developed only for the special case of binary polytrees \citep{fagiuoli1998a}, for which CU and PU coincides. 

\emph{*** This last part of the section has to be finished ***} Most of the literature about inference in CN is focused on approximate procedures for PU \cite{xxx}. Two remarkable exception are the early attempt of \cite{rocha} and a more recent paper by \cite{maua} where CU by VE is considered. In the first case, the convexification step, which is crucial to keep a bound on the exponential explosion on the number of extreme points, is performed. Unfortunately, as we show later, this induces an outer approximation. To avoid the outer approximation and keep the inferences exact, the second paper assumes the models extensively specified. In the present paper, we stick on the second approach, but assuming that the original model is separately specified. We therefore develop a number of online procedure which allow to perform the algebraic operations required by inference without xxx. Moreover, we suggest to perform convexification only on the space of single variables xxx.

The paper is organised as follows.
