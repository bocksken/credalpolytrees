\section{Introduction}\label{sec:intro}
Bayesian networks \citep[BNs]{2009:koller} are the most popular class of multivariate probabilistic models based on directed graphs. These tools are subject of intense theoretical investigation and their applications are ubiquitous, e.g., \cite{landuyt2013,de2013}. A common inference task with BNs, called \emph{updating}, consists in the computation of the probabilities of a variable of interest possibly given an instantiation of some other variables. BNs updating is a NP-hard task \citep{cooper1990}, which might become tractable under some additional assumption. If the standard \emph{variable elimination} \citep[VE]{kohlas} scheme is adopted, both time and space complexity are exponential in the \emph{treewidth}, which roughly describes how much a directed graph is similar to a singly-connected model (also called a polytree). Efficient updating can be therefore achieved (among others) in polytree-shaped BNs. This is basically equivalent to Pearl's message-propagation in polytrees \cite{pearl}.

Credal networks \citep[CNs]{cozman2000a} are a generalization of BNs based on the imprecise probability theory \citep{walley1991}. As well as a BN is specified by \emph{local} (w.r.t to the graph) probability mass functions, in a CN these are replaced by convex sets of them. We consequently call \emph{credal set updating} the task of  computing the credal set made of all the posterior mass functions for the queried variable given the evidence computed for all the BNs consistent with the CN specification. Here we call this task credal-set updating, while most of the authors intend updating as the computation of the lower and upper bounds of the posterior probability with respect to the CN specification. To avoid confusion, we call this latter task \emph{probabilistic updating} (PU). 

Being generalisations of their Bayesian counterpart, both tasks are NP-hard. Yet, unlike the case of BNs, PU remains NP-hard even in polytree-shaped CNs \citep{maua2014}, while an efficient scheme has been developed only for the special case of polytrees with all the variables binary \citep{fagiuoli1998a}. Note that for binary queries PU and CU coincides.\footnote{A credal set over a binary variable cannot have more thant two xxx.}

Most of the literature about inference in CN is focused on approximate procedures for PU \cite{xxx}. Two remarkable exception are the early attempt of \cite{rocha} and a more recent paper by \cite{maua} where CU by VE is considered. In the first case, the convexification step, which is crucial to keep a bound on the exponential explosion on the number of extreme points, is performed. Unfortunately, as we show later, this induces an outer approximation. To avoid the outer approximation and keep the inferences exact, the second paper assumes the models extensively specified. In the present paper, we stick on the second approach, but assuming that the original model is separately specified. We therefore develop a number of online procedure which allow to perform the algebraic operations required by inference without xxx. Moreover, we suggest to perform convexification only on the space of single variables xxx.
